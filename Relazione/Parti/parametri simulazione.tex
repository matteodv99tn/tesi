\chapter{Introduzione all'approccio SPICE}
	
	A livello accademico è stato descritto il principio di funzionamento dei \textit{MOSFET}, ossia dei transistor che utilizzano l'effetto di carico che si instaura tra uno substrato semiconduttivo e un metallo ossidato per movimentare delle cariche elettriche. In base al drogaggio dei terminali di \textit{source} e \textit{drain}, complementare a quello di \textit{bulk}, è possibile suddividere i transitori in due famiglie: gli \textit{n-MOS} (drogaggio di tipo $n$) e i \textit{p-MOS} (drogaggio di tipo $p$). In particolare la relazione statica che lega la corrente che scorre tra i terminali di drain e source è funzione sia della differenza di tensione $V_{gs}$ tra \textit{gate} e source, ma anche alla differenza di tensione $\Vds$ tra drain e source:
	\begin{equation} \label{eq:relstatica}
		I = K_n \frac W L \left[ \big(\Vgs - \Vtn\big)\Vds - \frac{\Vds^2}{2} \right]
	\end{equation} 
	In questa relazione è possibile osservare la presenza di 3 parametri fondamentali a determinare il comportamento del transistor: la \textit{conducibilità intrinseca} $K_n$, proprietà caratteristica del semiconduttore utilizzato per il bulk, e le dimensioni caratteristiche $W$ (larghezza) e $L$ (lunghezza) del canale conduttivo. Nella caratteristica statica fondamentale è anche la \textit{tensione di soglia} $\Vtn$ dipendente sia dalla costituzione del transistor, sia dalla differenza di tensione $V_{bs}$ tra bulk e source. 
	
	Il modello presentato in equazione \ref{eq:relstatica} è in realtà una versione approssimata della caratteristica di trasferimento reale di un transistor MOS e trascura molti fenomeni elettro-magnetici che nella realtà dovrebbero essere considerati; esso può essere utile a livello didattico per concepire il funzionamento di alcuni circuiti semplici, tuttavia per problemi più complessi un approccio analitico approssimato può portare a risultati fuorvianti. \\
	Un approccio simulativo è infatti più indicato per poter analizzare le prestazioni di circuiti più complessi in quanto a prova di errori (una volta che ci si è assicurati di aver implementato correttamente gli schematici) e permette di considerare effetti elettro-magnetici che analiticamente sarebbe difficile da studiare.
	
	\vspace{3mm}
	
	In ambito elettronico per effettuare delle simulazioni numeriche di circuiti si utilizzano i software cosiddetti \textit{SPICE} (acronimo di \textit{Simulation Program with Integrated Circuit Emphasis}); in particolare tra le numerose soluzioni disponibili sul mercato nel proseguimento del seguente testo verrà utilizzato il software gratuito \texttt{XScheme} \cite{xschem} per la realizzazione degli schematici che verranno simulati tramite l'applicativo \texttt{ngspice} \cite{ngspice}.
	
\section{Parametri di simulazione}
	Per poter effettuare delle simulazioni è necessario fornire al software una raccolta con le informazioni da utilizzare per modellare il transistor, ossia è necessario specificare tutti i parametri che possono essere sia geometrici, ma anche legati alle proprietà dei materiali. 
	
	Facendo diretto riferimento ai parametri presenti nell'equazione \ref{eq:relstatica} per un transistor è necessario in primo luogo indicare la conducibilità intrinseca \texttt{Kp} $[A/V^2]$, la lunghezza \texttt{L} $[m]$  e la larghezza \texttt{W} $[m]$  del canale conduttivo. Altri parametri geometrici che possono essere utilizzati per migliorare l'analisi è indicare sia perimetro che area per il terminale di drain (parametri \texttt{PD} $[m]$ e \texttt{AD} $[m^2]$) e il terminale source (parametri \texttt{PS} e \texttt{AS}).
	
	Come parametri funzionali per il calcolo della caratteristica statica dei MOSFET si menziona la tensione di soglia, modellata tramite il parametro \texttt{Vto} $[V]$. L'effetto body, dovuto alla differenza di tensione tra bulk e source, richiede invece di specificare il relativo coefficiente \texttt{Gamma} $\left[V^{0.5}\right]$ e il coefficiente superficiale \texttt{Phi} $[V]$. Come ultimo parametro di un transistor si menziona il coefficiente di modulazione di lunghezza di canale \texttt{Lambda} $\left[V^{-1}\right]$.
	
	\begin{table}[bht]
		
		\centering
		\begin{tabular}{ c c | p{2cm}  p{2cm}}
			& & \multicolumn{2}{c}{famiglia di transistor }  \\
			parametro & unità & n-MOS & p-MOS \\ \hline 
			\texttt{K} & $[A/V^2]$ & $50\cdot 10^{-6}$ & $20\cdot 10^{-6}$ \\
			\texttt{W} & $[m]$ & $50\cdot 10^{-6}$ & $20\cdot 10^{-6}$ \\
			\texttt{L} & $[m]$ & $50\cdot 10^{-6}$ & $20\cdot 10^{-6}$ \\
		\end{tabular}
		\caption{parametri di simulazioni utilizzati nel seguente documenti; i dati sono basati su transistor \textbf{ALTRE INFORMAZIONI}}
		
		
	\end{table}


	Un componente reale, in condizioni sia statiche, presenta delle perdite di corrente sia tra drain e source, sia tra gate e source, nel cosiddetto fenomeno della \textit{current leakage} (perdita di corrente) associato alle correnti parassite. Analizzando invece il comportamento dinamico del circuito è possibile osservare che i MOSFET presentano un'inerzia alla trasmissione di carica (rispetto ad ogni coppia di terminali): tali effetti di \textit{capacità parassite} possono essere modellate tramite l'inserimento nello schematico di capacità equivalenti. \\
	Nella pratica le relazioni che determinano correnti e capacità parassite sono complesse (equazioni fortemente non lineari) e dipendenti da molti parametri dei transistor stessi: non esiste dunque un modello univoco che può essere utilizzato per la simulazione dei circuiti (ad un livello di complessità simil-realistico), ma in generale ogni produttore mette a disposizione dei progettisti i loro modelli spice che possono dunque essere inclusi negli schematici per effettuare delle simulazioni più interessanti.
	
\section{\textit{Process Design Kit}: skywater}
	
	Il \textit{Process Design Kit}, spesso abbreviato dall'acronimo \textit{PDK}, è una suite di librerie e applicativi che permettono una progettazione corretta di un circuito integrato. In questi kit sono contenuti infatti tutti i modelli spice (sia dal modello lineare più semplice, sia a modelli del $4^\circ$ ordine più complessi) che possono essere utilizzati per le simulazioni, oltre che a una serie di informazioni che vincolano la progettazione per permettere di ottenere un prodotto che sia effettivamente utilizzabile nel mondo reale. Per esempio, oltre a tutte le informazioni riguardanti ingombri fisici, i PDK contengono le proprietà per simulare con maggior precisione le correnti e capacità parassite che si generano nel prodotto finito.
	
	\skywater, come dice il nome stesso, è dunque un PDK rilasciato pubblicamente frutto della collaborazione di Google con la fondazione Skywater; questo progetto, per come riportato dal team di sviluppo del PDK stesso, è ancora in fase sperimentale e dunque può non essere perfettamente accurato, tuttavia si osserva che lo stesso progetto deriva direttamente da PDK utilizzati da anni a livello professionale. \\
	L'idea alla base di questo progetto open source è quella di permettere a tutte le persone di progettare e prototipare circuiti integrati, permettendo la realizzazione pratica sfruttando il processo produttivo a $130 nm$ da \texttt{FOSSi} (\textit{Free and Open Source Silicon} components) \texttt{foundation} \cite{fossi}.
	
	
	Sfruttando la suite di software composta da \texttt{XSchem}, \texttt{ngspice} e \texttt{skywater PDK} è possibile realizzare degli schematici e dei circuiti che si avvicinano il più possibile a dei circuiti reali.
	
	\subsection*{Caratteristica statica}
	
	
	
	
	
	
	
	
	
	
	
	
	
	
	
	
	
	
	
	
	
	
	
	
	
	
	