\chapter{Preambolo}
	Spesso nell'utilizzo comune di dispositivi digitali si dà per scontato il funzionamento intrinseco degli stessi, pensando che il computer ragiona solo con valori binari \textit{0 ed 1}, \textit{tensione alta e tensione bassa}. Spesso si dimentica però che, come ogni oggetto, anche i componenti che costituiscono i nostri device tecnologici sono, in origine, dei componenti analogici.
	
	Lo scopo di questo documento è dunque quello di studiare come dei dispositivi analogici, in particolari i \textit{transistor MOS}, possono essere utilizzati in ambito digitale, evidenziandone dunque i limiti fisici e dinamici legati alla loro implementazione. L'obiettivo sarà dunque quello di descrivere i principali circuiti che sono posti alla base di ogni calcolatore digitale, come le porte logiche, fino ad arrivare all'analisi di circuiti come il sommatore e moltiplicatore binario in tecnologia \textit{c-MOS}.
	
	L'approccio utilizzato per analizzare il problema sarà più di tipo simulativo: gli schematici sono realizzati tramite il software open source \texttt{XSchem} \cite{xschem}, mentre le simulazioni vengono effettuate mediante l'utilizzo del simulatore \texttt{ngspice} \cite{ngspice}. Per rendere tutto l'approccio il più reale ed applicativo possibile verranno utilizzati i modelli spice rilasciati pubblicamente mediante il progetto open source \texttt{google-skywater PDK} \cite{skywater}.