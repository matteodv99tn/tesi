\documentclass[10 pt,letterpaper,twoside,openright]{book}
\usepackage[utf8]{inputenc}
\usepackage[T1]{fontenc}
\usepackage[italian]{babel}
\usepackage{amsmath}
\usepackage{amsfonts}
\usepackage{amssymb}
\usepackage{graphicx}
\usepackage[margin=1cm]{caption}
\usepackage{hyperref}
\usepackage[left=4.5cm, width=14.00cm, height=21.00cm]{geometry}
\usepackage{ragged2e}

\usepackage{biblatex} 
\addbibresource{bibliografia.bib} 


\usepackage{subcaption}
\usepackage{float}


\captionsetup{font = {it, small}, labelfont={bf}}

\newcommand{\Vgs}{V_{gs}}
\newcommand{\Vds}{V_{ds}}
\newcommand{\Vtn}{V_{tn}}
\newcommand{\Vtnz}{V_{tn,0}}
\newcommand{\Vout}{V_{out}}


\newcommand{\skywater}{\texttt{skywater PDK} \cite{skywater}}

\title{Tesi di Laurea}
\author{Matteo Dalle Vedove}
\date{\today}


\begin{document}
	\frontmatter
	
	\thispagestyle{empty}
	\begin{center}
		\includegraphics[width=6cm]{Immagini/logo}
		
		\vspace{2cm}
		{\Large Corso di Laurea Triennale in \\ Ingegneria Industriale}
		
		\vspace{6mm}
		
		\rule{5cm}{0.5pt}	
		\vspace{6mm}
		
		{\LARGE Prova Finale
		
		\vspace{6mm}
		
		\textbf{Analisi di circuiti ad applicazione digitale in tecnologia c-MOS}
		\vspace{6mm}
		
		\rule{5cm}{0.5pt}	
		 }
		
	\end{center}
	{ \large
	\vspace{0.5cm} \noindent
	Relatore:\\ Prof. Dalla Betta Gian-Franco
	
	\vspace{1.5cm} \noindent
	Co-relatore: \\ Dott. Parmesan Luca
	
	\vspace{1.5cm} \noindent
	Studente: \\
	Dalle Vedove Matteo, 203063
	
	\vspace{1.5cm}
	\begin{center}
		Anno Accademico 2020-2021
	\end{center}
	
	}

	\chapter*{Ringraziamenti}
	\begin{flushright}
		\textit{Ringrazio Vira, mia fonte di ispirazione che mi ha sempre spinto a dare il meglio di me,\\ supportandomi e sopportandomi nonostante le mille difficoltà. }
	\end{flushright}

	\begin{flushright}
		\textit{Ringrazio i miei genitori, i nonni e tutti i parenti che hanno sempre creduto in me e hanno contribuito a rendere possibile questo percorso.}
	\end{flushright}

	\section*{Introduzione}
	Spesso nell'utilizzo comune di dispositivi digitali si da per scontato il funzionamento intrinseco degli stessi, pensando che il computer ragiona solo con valori binario \textit{0 ed 1}, \textit{tensione alta e tensione bassa}. Spesso si dimentica però che, come ogni oggetto, anche i componenti che costituiscono i nostri device tecnologici sono, in origine, dei componenti analogici.
	
	Lo scopo di questo documento è dunque quello di studiare come dei dispositivi analogici, in particolari i \textit{transistor MOS}, possono essere utilizzati in ambito digitale, evidenziandone dunque i limiti fisici e dinamici legati alla loro implementazione. L'obiettivo sarà dunque quello di descrivere i principali circuiti che sono posti alla base di ogni calcolatore digitale, come le porte logiche, fino ad arrivare all'analisi di circuiti come il sommatore e moltiplicatore binario in tecnologia \textit{c-mos}.
	
	L'approccio utilizzato per analizzare il problema sarà più di tipo simulativo utilizzando il software freeware \texttt{LTspice} \cite{ltspice} supportato dal fornitore di componentistica elettronica \texttt{Analog Devices}	.
	\tableofcontents
	
	
	
	\mainmatter
	
	\section{Introduzione all'approccio SPICE}
	
	A livello accademico è stato descritto il principio di funzionamento dei \textit{MOSFET}, ossia dei transistor che utilizzano l'effetto di carico che si instaura tra uno substrato semiconduttivo e un metallo ossidato per movimentare delle cariche elettriche. In base al drogaggio dei terminali di \textit{source} e \textit{drain}, complementare a quello di \textit{bulk}, è possibile suddividere i transitori in due famiglie: gli \textit{n-MOS} (drogaggio di tipo $n$) e i \textit{p-MOS} (drogaggio di tipo $p$). In particolare la relazione statica che lega la corrente che scorre tra i terminali di drain e source è funzione sia della differenza di tensione $V_{gs}$ tra \textit{gate} e source, ma anche alla differenza di tensione $\Vds$ tra drain e source:
	\begin{equation} \label{eq:relstatica}
		I = K_n \frac W L \left[ \big(\Vgs - \Vtn\big)\Vds - \frac{\Vds^2}{2} \right]
	\end{equation} 
	In questa relazione è possibile osservare la presenza di 3 parametri fondamentali a determinare il comportamento del transistor: la \textit{conducibilità intrinseca} $K_n$, proprietà caratteristica del semiconduttore utilizzato per il bulk, e le dimensioni caratteristiche $W$ (larghezza) e $L$ (lunghezza) del canale conduttivo. Nella caratteristica statica fondamentale è anche la \textit{tensione di soglia} $\Vtn$ dipendente sia dalla costituzione del transistor, sia dalla differenza di tensione $V_{bs}$ tra bulk e source. 
	
	Il modello presentato in equazione \ref{eq:relstatica} è in realtà una versione approssimata della caratteristica di trasferimento reale di un transistor MOS e trascura molti fenomeni elettro-magnetici che nella realtà dovrebbero essere considerati; esso può essere utile a livello didattico per concepire il funzionamento di alcuni circuiti semplici, tuttavia per problemi più complessi un approccio analitico approssimato può portare a risultati fuorvianti. \\
	Un approccio simulativo è infatti più indicato per poter analizzare le prestazioni di circuiti più complessi in quanto a prova di errori (una volta che ci si è assicurati di aver implementato correttamente gli schematici) e permette di considerare effetti elettro-magnetici che analiticamente sarebbe difficile da studiare.
	
	\vspace{3mm}
	
	In ambito elettronico per effettuare delle simulazioni numeriche di circuiti si utilizzano i software cosiddetti \textit{SPICE} (acronimo di \textit{Simulation Program with Integrated Circuit Emphasis}); in particolare tra le numerose soluzioni disponibili sul mercato nel proseguimento del seguente testo si farà riferimento a \texttt{LTspice XVII} \cite{ltspice}, software gratuito attualmente mantenuto dal produttore \texttt{Analog Devices}, aggiornato alla versione \texttt{17.0.27.0}.
	
\subsection{Parametri di simulazione}
	Per poter effettuare delle simulazioni è necessario fornire al software una raccolta con le informazioni da utilizzare per modellare il transistor, ossia è necessario specificare tutti i parametri che possono essere sia geometrici, ma anche legati alle proprietà dei materiali. 
	
	Facendo diretto riferimento ai parametri presenti nell'equazione \ref{eq:relstatica} per un transistor è necessario in primo luogo indicare la conducibilità intrinseca \texttt{K} $[A/V^2]$, la lunghezza \texttt{L} $[m]$  e la larghezza \texttt{W} $[m]$  del canale conduttivo. Altri parametri geometrici che possono essere utilizzati per migliorare l'analisi è indicare sia perimetro che area per il terminale di drain (parametri \texttt{PD} $[m]$ e \texttt{AD} $[m^2]$) e il terminale source (parametri \texttt{PS} e \texttt{AS}).
	
	
	\begin{table}[p]
		
		\centering
		\begin{tabular}{ c c | p{2cm}  p{2cm}}
			& & \multicolumn{2}{c}{famiglia di transistor }  \\
			parametro & unità & n-MOS & p-MOS \\ \hline 
			\texttt{K} & $[A/V^2]$ & $50\cdot 10^{-6}$ & $20\cdot 10^{-6}$ \\
			\texttt{W} & $[m]$ & $50\cdot 10^{-6}$ & $20\cdot 10^{-6}$ \\
			\texttt{L} & $[m]$ & $50\cdot 10^{-6}$ & $20\cdot 10^{-6}$ \\
		\end{tabular}
		\caption{parametri di simulazioni utilizzati nel seguente documenti; i dati sono basati su transistor \textbf{ALTRE INFORMAZIONI}}
		
	\end{table}
	
	\chapter{Porte logiche in tecnologia c-MOS}
\subsection{NOT gate}


\subsection{AND gate}


\subsection{OR gate}



\subsection{NOR gate}



\subsection{NAND gate}

	
	\section{Elementi di memoria}
\subsection{SR latch}
\subsection{RAM dinamica}
\subsection{RAM statica}
	
	\section{Circuiti logici}

\subsection{Sommatore}


\subsection{Moltiplicatore}
	
	\chapter{Conclusioni}
	
	Lo scopo di questa tesi è stata quella di approfondire alcuni circuiti ad applicazione digitale con particolare attenzione all'implementazione in tecnologia c-mos. L'approccio prettamente descrittivo del documento, con supporto di simulazioni spice, permette di avere una visione complessiva di come avviene la conversione del pensiero da analogico a digitale facendo riferimento ad alcune particolari reti logiche.
	
	L'analisi di circuiti digitali comprende un'elevata quantità di informazioni e varianti che è materialmente impossibile descrivere in un unico documento, tuttavia l'obbiettivo generale è quello di acquisire un'approccio più che concreto per comprendere il funzionamento di reti logiche che compongono tutti i dispositivi tecnologici che ci circondano.
	
	La redazione di questo documento ha inoltre permesso a me personalmente di confrontarmi con nuovi software destinati a un pubblico appassionato alla progettazione di circuiti integrati, permettendo così un'interazione con il mondo non solo prettamente accademico, ma anche improntato ad un'utilità produttiva.
	
	\backmatter
	
	% \chapter{Schematici}
	
	\chapter{Bibliografia e sitografia}
	\printbibliography[heading=none]
	
	\noindent
	Per la stesura del seguente documento sono stati consultati:
	\begin{itemize}
		\item \textit{CMOS Memory}, Giuseppe Ianaccone, corso di Electronic System (anno 2016), Università di Pisa, \url{http://www.iannaccone.org/ES2016/}
	\end{itemize}
	
	
	
	
		
\end{document}