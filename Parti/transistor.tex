\section{MOSFET}
	
	I transistori \textit{MOSFET}, acronimo di \textit{metal-oxide semiconductor field-effect transistor}, sono dei dispositivi analogici che sfruttano l'effetto di carico per veicolare una quantità di corrente prestabilita tra i suoi terminali. In particolare i dispositivi sono realizzati tramite il contatto di uno strato metallico ossidato, detto \textit{gate} $G$, ed un substrato di semiconduttore (in genere silicio), detto \textit{bulk} $B$.
	
	Un substrato di semiconduttore con drogaggio $p$ dà origine ai cosiddetti \textit{n-MOS}: i due terminali attraverso i quali si potrà verificare successivamente il trasferimento di carica sono realizzati in semiconduttore a drogaggio $n$ e prendono il nome di \textit{source} $S$ e \textit{drain} $D$. Imponendo una differenza di potenziale elettrostatico tra gate e bulk è possibile controllare il canale conduttivo che regola il passaggio di elettroni, e dunque di corrente elettrica, che avverrà sempre dal drain al source.
	
	Invertendo il drogaggio tra bulk (ponendolo a $n$) e terminali di source e drain (ponendoli a $p$) è possibile trovare il transistor duale all'n-MOS, ossia il \textit{p-MOS}. In \textbf{FIGURA} è possibile osservare la rappresentazione circuitale sia per i transistor a canale $n$ che per transistor a canale $p$.
	
	\textbf{FARE FIGURA DEGLI SCHEMI CIRCUITALI}
	
\subsection{Relazioni statiche dei transistor}
	In precedenza si è affermato che regolando la differenza di tensione $\Vgs$ tra gate e bulk è possibile controllare il canale conduttivo, e dunque il passaggio di corrente; per stabilire univocamente l'equazione che determina la corrente $I$ che fluisce tra drain e source è necessario tuttavia conoscere anche la differenza di tensione $\Vds$ tra il terminale di drain e di source: la relazione statica che determina la corrente passante tra i due terminali vale dunque
	\begin{equation}
		I = K_n \frac W L \left[ \big(\Vgs - \Vtn\big)\Vds - \frac {\Vds^2}2 \right]
	\end{equation}
	
	In questa equazione il coefficiente $K_n$ rappresenta la \textit{conducibilità intrinseca} del materiale, misurata tendenzialmente in $\mu A/V^2$, il termine $W/L$ rappresenta il rapporto tra larghezza e lunghezza del canale conduttivo e $\Vtn$ è la \textit{tensione di soglia} (dipendente dalla tensione di soglia a terminali cortocircuitati $\Vtnz$ e dall'effetto body). In particolari per tensioni $\Vgs$ inferiori alla tensione di soglia $\Vtn$ si osserva che la corrente $I$ tra drain e source è sempre nulla.